\documentclass[]{article}
\usepackage{lmodern}
\usepackage{amssymb,amsmath}
\usepackage{ifxetex,ifluatex}
\usepackage{fixltx2e} % provides \textsubscript
\ifnum 0\ifxetex 1\fi\ifluatex 1\fi=0 % if pdftex
  \usepackage[T1]{fontenc}
  \usepackage[utf8]{inputenc}
\else % if luatex or xelatex
  \ifxetex
    \usepackage{mathspec}
  \else
    \usepackage{fontspec}
  \fi
  \defaultfontfeatures{Ligatures=TeX,Scale=MatchLowercase}
\fi
% use upquote if available, for straight quotes in verbatim environments
\IfFileExists{upquote.sty}{\usepackage{upquote}}{}
% use microtype if available
\IfFileExists{microtype.sty}{%
\usepackage{microtype}
\UseMicrotypeSet[protrusion]{basicmath} % disable protrusion for tt fonts
}{}
\usepackage[margin=1in]{geometry}
\usepackage{hyperref}
\hypersetup{unicode=true,
            pdftitle={Lab 2, Group B9},
            pdfauthor={Anton Gefvert, Richard Friberg, Ruben Hillborg},
            pdfborder={0 0 0},
            breaklinks=true}
\urlstyle{same}  % don't use monospace font for urls
\usepackage{color}
\usepackage{fancyvrb}
\newcommand{\VerbBar}{|}
\newcommand{\VERB}{\Verb[commandchars=\\\{\}]}
\DefineVerbatimEnvironment{Highlighting}{Verbatim}{commandchars=\\\{\}}
% Add ',fontsize=\small' for more characters per line
\usepackage{framed}
\definecolor{shadecolor}{RGB}{248,248,248}
\newenvironment{Shaded}{\begin{snugshade}}{\end{snugshade}}
\newcommand{\KeywordTok}[1]{\textcolor[rgb]{0.13,0.29,0.53}{\textbf{#1}}}
\newcommand{\DataTypeTok}[1]{\textcolor[rgb]{0.13,0.29,0.53}{#1}}
\newcommand{\DecValTok}[1]{\textcolor[rgb]{0.00,0.00,0.81}{#1}}
\newcommand{\BaseNTok}[1]{\textcolor[rgb]{0.00,0.00,0.81}{#1}}
\newcommand{\FloatTok}[1]{\textcolor[rgb]{0.00,0.00,0.81}{#1}}
\newcommand{\ConstantTok}[1]{\textcolor[rgb]{0.00,0.00,0.00}{#1}}
\newcommand{\CharTok}[1]{\textcolor[rgb]{0.31,0.60,0.02}{#1}}
\newcommand{\SpecialCharTok}[1]{\textcolor[rgb]{0.00,0.00,0.00}{#1}}
\newcommand{\StringTok}[1]{\textcolor[rgb]{0.31,0.60,0.02}{#1}}
\newcommand{\VerbatimStringTok}[1]{\textcolor[rgb]{0.31,0.60,0.02}{#1}}
\newcommand{\SpecialStringTok}[1]{\textcolor[rgb]{0.31,0.60,0.02}{#1}}
\newcommand{\ImportTok}[1]{#1}
\newcommand{\CommentTok}[1]{\textcolor[rgb]{0.56,0.35,0.01}{\textit{#1}}}
\newcommand{\DocumentationTok}[1]{\textcolor[rgb]{0.56,0.35,0.01}{\textbf{\textit{#1}}}}
\newcommand{\AnnotationTok}[1]{\textcolor[rgb]{0.56,0.35,0.01}{\textbf{\textit{#1}}}}
\newcommand{\CommentVarTok}[1]{\textcolor[rgb]{0.56,0.35,0.01}{\textbf{\textit{#1}}}}
\newcommand{\OtherTok}[1]{\textcolor[rgb]{0.56,0.35,0.01}{#1}}
\newcommand{\FunctionTok}[1]{\textcolor[rgb]{0.00,0.00,0.00}{#1}}
\newcommand{\VariableTok}[1]{\textcolor[rgb]{0.00,0.00,0.00}{#1}}
\newcommand{\ControlFlowTok}[1]{\textcolor[rgb]{0.13,0.29,0.53}{\textbf{#1}}}
\newcommand{\OperatorTok}[1]{\textcolor[rgb]{0.81,0.36,0.00}{\textbf{#1}}}
\newcommand{\BuiltInTok}[1]{#1}
\newcommand{\ExtensionTok}[1]{#1}
\newcommand{\PreprocessorTok}[1]{\textcolor[rgb]{0.56,0.35,0.01}{\textit{#1}}}
\newcommand{\AttributeTok}[1]{\textcolor[rgb]{0.77,0.63,0.00}{#1}}
\newcommand{\RegionMarkerTok}[1]{#1}
\newcommand{\InformationTok}[1]{\textcolor[rgb]{0.56,0.35,0.01}{\textbf{\textit{#1}}}}
\newcommand{\WarningTok}[1]{\textcolor[rgb]{0.56,0.35,0.01}{\textbf{\textit{#1}}}}
\newcommand{\AlertTok}[1]{\textcolor[rgb]{0.94,0.16,0.16}{#1}}
\newcommand{\ErrorTok}[1]{\textcolor[rgb]{0.64,0.00,0.00}{\textbf{#1}}}
\newcommand{\NormalTok}[1]{#1}
\usepackage{graphicx,grffile}
\makeatletter
\def\maxwidth{\ifdim\Gin@nat@width>\linewidth\linewidth\else\Gin@nat@width\fi}
\def\maxheight{\ifdim\Gin@nat@height>\textheight\textheight\else\Gin@nat@height\fi}
\makeatother
% Scale images if necessary, so that they will not overflow the page
% margins by default, and it is still possible to overwrite the defaults
% using explicit options in \includegraphics[width, height, ...]{}
\setkeys{Gin}{width=\maxwidth,height=\maxheight,keepaspectratio}
\IfFileExists{parskip.sty}{%
\usepackage{parskip}
}{% else
\setlength{\parindent}{0pt}
\setlength{\parskip}{6pt plus 2pt minus 1pt}
}
\setlength{\emergencystretch}{3em}  % prevent overfull lines
\providecommand{\tightlist}{%
  \setlength{\itemsep}{0pt}\setlength{\parskip}{0pt}}
\setcounter{secnumdepth}{0}
% Redefines (sub)paragraphs to behave more like sections
\ifx\paragraph\undefined\else
\let\oldparagraph\paragraph
\renewcommand{\paragraph}[1]{\oldparagraph{#1}\mbox{}}
\fi
\ifx\subparagraph\undefined\else
\let\oldsubparagraph\subparagraph
\renewcommand{\subparagraph}[1]{\oldsubparagraph{#1}\mbox{}}
\fi

%%% Use protect on footnotes to avoid problems with footnotes in titles
\let\rmarkdownfootnote\footnote%
\def\footnote{\protect\rmarkdownfootnote}

%%% Change title format to be more compact
\usepackage{titling}

% Create subtitle command for use in maketitle
\newcommand{\subtitle}[1]{
  \posttitle{
    \begin{center}\large#1\end{center}
    }
}

\setlength{\droptitle}{-2em}

  \title{Lab 2, Group B9}
    \pretitle{\vspace{\droptitle}\centering\huge}
  \posttitle{\par}
    \author{Anton Gefvert, Richard Friberg, Ruben Hillborg}
    \preauthor{\centering\large\emph}
  \postauthor{\par}
      \predate{\centering\large\emph}
  \postdate{\par}
    \date{12/10/2018}


\begin{document}
\maketitle

\section{Contributions}\label{contributions}

\begin{itemize}
\tightlist
\item
  Assignment 1: Anton Gefvert
\item
  Assignment 2:
\item
  Assignment 3:
\end{itemize}

\section{Assignment 1}\label{assignment-1}

\subsection{1.1}\label{section}

\begin{figure}
\centering
\includegraphics{Group_B9_files/figure-latex/load_crabs-1.pdf}
\caption{\label{fig:crabs} Carapace Length vs Rear Width classified by
sex}
\end{figure}

Looking at Figure \ref{fig:crabs}, we see that drawing a line to
discriminate between male and female crabs on these data would be very
feasable, thus classifying by linear discriminant analysis would be easy
in this case.

\subsection{1.2}\label{section-1}

If we compare Figure \ref{fig:crabs_lda} to Figure \ref{fig:crabs}, we
see that they are very similar. There are some differences when carapace
length is small, e.g the bottom left point is actually female. but is
classified as male with the lda. There are also some differences when
you look at the points which are kind of overlapping each other (some
males classfied as females and vice versa). We notice though, that the
bigger the carapace length is, the more accurate the lda model is!

\begin{figure}
\centering
\includegraphics{Group_B9_files/figure-latex/lda_crabs-1.pdf}
\caption{\label{fig:crabs_lda} Carapace Length vs Rear Width classified
by sex using lda}
\end{figure}

When using this lda model we get a missclassification rate of \(4.5\%\),
this indicates that the model is very well fitted to this problem.

\subsection{1.3}\label{section-2}

\begin{figure}
\centering
\includegraphics{Group_B9_files/figure-latex/lda_crabs_prior-1.pdf}
\caption{\label{fig:crabs_lda_prior} Carapace Length vs Rear Width
classified by sex using lda (with prior)}
\end{figure}

As seen in Figure \ref{fig:crabs_lda_prior},when using
\(p(Male)=0.9, p(Female)=0.1\) as a prior, we get (compared to Figure
\ref{fig:crabs_lda_prior}), as expected when weighting the male sex
higher, a more male dominated graph. This can especially be seen in the
lower bounds of carapace length and the values that are pretty close in
between the two separation (e.g.~the value around \(CL=38\) and
\(RW=15\) is female in Figure \ref{fig:crabs_lda} and male in Figure
\ref{fig:crabs_lda_prior}).

If we look at the missclassification rate when using the prior we get a
missclassification rate of \(8\%\), this is still very good, but almost
twice as much when not using prior (or rather using a prior
\(p(Male)=p(Female)=0.5\))

\subsection{1.4}\label{section-3}

\begin{figure}
\centering
\includegraphics{Group_B9_files/figure-latex/crab_glm_plot-1.pdf}
\caption{\label{fig:crabs_glm} Carapace Length vs Rear Width classified
by sex using glm}
\end{figure}

If we look at Figure \ref{fig:crabs_glm} we see that it looks very
similar to all the other figures. It looks very much like Figure
\ref{fig:crabs_lda}, and is more keen to classify uncertain values as
female than when using lda with prior.

Missclassification rate for this method is \(3.5\%\), so slightly better
than the other methods.

The equation for the decision boundary (green line in Figure
\ref{fig:crabs_glm}) is \(RW = 0.369 CL + 1.08\)

\section{Assignment 2}\label{assignment-2}

\section{Assignment 4 - Principal
components}\label{assignment-4---principal-components}

\subsection{1 Standard PCA}\label{standard-pca}

\begin{verbatim}
##   [1] 1.489914e-02 9.998545e-04 2.954195e-05 1.608532e-05 1.091077e-05
##   [6] 3.939315e-06 1.414911e-06 4.981545e-07 4.262849e-07 2.577774e-07
##  [11] 2.080001e-07 1.587511e-07 1.425823e-07 1.126727e-07 7.232246e-08
##  [16] 6.878939e-08 5.307373e-08 4.373598e-08 3.975200e-08 3.627181e-08
##  [21] 3.473207e-08 2.838554e-08 2.750156e-08 2.356802e-08 2.057859e-08
##  [26] 1.921151e-08 1.772579e-08 1.719151e-08 1.546958e-08 1.450458e-08
##  [31] 1.349010e-08 1.229577e-08 1.210005e-08 1.144210e-08 1.068630e-08
##  [36] 1.046807e-08 9.148433e-09 8.884085e-09 8.567593e-09 8.126130e-09
##  [41] 7.768325e-09 7.271742e-09 7.005011e-09 6.462762e-09 6.415715e-09
##  [46] 6.123419e-09 5.705293e-09 5.634860e-09 5.489343e-09 5.237779e-09
##  [51] 5.146764e-09 4.927885e-09 4.683481e-09 4.541157e-09 4.483382e-09
##  [56] 4.334378e-09 4.102898e-09 3.859924e-09 3.754631e-09 3.735784e-09
##  [61] 3.569046e-09 3.458093e-09 3.357414e-09 3.280258e-09 3.117910e-09
##  [66] 3.077594e-09 2.965870e-09 2.877830e-09 2.821708e-09 2.689767e-09
##  [71] 2.553543e-09 2.451608e-09 2.443009e-09 2.351475e-09 2.323987e-09
##  [76] 2.261444e-09 2.210244e-09 2.146417e-09 2.044353e-09 1.957622e-09
##  [81] 1.932558e-09 1.871133e-09 1.864625e-09 1.752181e-09 1.698602e-09
##  [86] 1.676920e-09 1.656262e-09 1.581932e-09 1.557666e-09 1.467634e-09
##  [91] 1.410370e-09 1.380420e-09 1.347822e-09 1.336314e-09 1.236098e-09
##  [96] 1.209134e-09 1.179186e-09 1.123574e-09 1.068961e-09 9.985917e-10
## [101] 9.858015e-10 9.526850e-10 8.980246e-10 8.728980e-10 8.288593e-10
## [106] 8.144341e-10 7.545504e-10 7.293165e-10 7.199373e-10 6.733103e-10
## [111] 6.580809e-10 6.168986e-10 6.099031e-10 5.763810e-10 5.525311e-10
## [116] 5.421167e-10 5.266189e-10 5.093443e-10 4.827224e-10 4.463711e-10
## [121] 4.127067e-10 3.895925e-10 3.786334e-10 3.495125e-10 3.033627e-10
## [126] 2.675822e-10
\end{verbatim}

\begin{verbatim}
##   [1] "93.332" "6.263"  "0.185"  "0.101"  "0.068"  "0.025"  "0.009" 
##   [8] "0.003"  "0.003"  "0.002"  "0.001"  "0.001"  "0.001"  "0.001" 
##  [15] "0.000"  "0.000"  "0.000"  "0.000"  "0.000"  "0.000"  "0.000" 
##  [22] "0.000"  "0.000"  "0.000"  "0.000"  "0.000"  "0.000"  "0.000" 
##  [29] "0.000"  "0.000"  "0.000"  "0.000"  "0.000"  "0.000"  "0.000" 
##  [36] "0.000"  "0.000"  "0.000"  "0.000"  "0.000"  "0.000"  "0.000" 
##  [43] "0.000"  "0.000"  "0.000"  "0.000"  "0.000"  "0.000"  "0.000" 
##  [50] "0.000"  "0.000"  "0.000"  "0.000"  "0.000"  "0.000"  "0.000" 
##  [57] "0.000"  "0.000"  "0.000"  "0.000"  "0.000"  "0.000"  "0.000" 
##  [64] "0.000"  "0.000"  "0.000"  "0.000"  "0.000"  "0.000"  "0.000" 
##  [71] "0.000"  "0.000"  "0.000"  "0.000"  "0.000"  "0.000"  "0.000" 
##  [78] "0.000"  "0.000"  "0.000"  "0.000"  "0.000"  "0.000"  "0.000" 
##  [85] "0.000"  "0.000"  "0.000"  "0.000"  "0.000"  "0.000"  "0.000" 
##  [92] "0.000"  "0.000"  "0.000"  "0.000"  "0.000"  "0.000"  "0.000" 
##  [99] "0.000"  "0.000"  "0.000"  "0.000"  "0.000"  "0.000"  "0.000" 
## [106] "0.000"  "0.000"  "0.000"  "0.000"  "0.000"  "0.000"  "0.000" 
## [113] "0.000"  "0.000"  "0.000"  "0.000"  "0.000"  "0.000"  "0.000" 
## [120] "0.000"  "0.000"  "0.000"  "0.000"  "0.000"  "0.000"  "0.000"
\end{verbatim}

\begin{verbatim}
## [1] "Feature 1 & 2 gives us a percentage of 99.596"
\end{verbatim}

\includegraphics{Group_B9_files/figure-latex/unnamed-chunk-1-1.pdf}
\includegraphics{Group_B9_files/figure-latex/unnamed-chunk-1-2.pdf}

In the plot with all the points we can see some outliners that can be
viewed as being `unusual', as the are not part of the sort of big cloud
of points.

\subsection{2 Trace plots}\label{trace-plots}

\includegraphics{Group_B9_files/figure-latex/unnamed-chunk-2-1.pdf}
\includegraphics{Group_B9_files/figure-latex/unnamed-chunk-2-2.pdf}

Yes, PC2 can be explained by mainly a few original features. The
features around the peak at index 124 contributes a lot to that
component. The range in PC1 is from about 0.08 to 0.110 wich is pretty
narrow so no one of the features really stand out.

\subsection{3 Independent Component
Analysis}\label{independent-component-analysis}

\paragraph{A}\label{a}

\includegraphics{Group_B9_files/figure-latex/unnamed-chunk-3-1.pdf}
\includegraphics{Group_B9_files/figure-latex/unnamed-chunk-3-2.pdf}

The measures representing is the loadings, which we can see as the plots
look like the ones in 4.2 only mirrored.

\paragraph{B}\label{b}

\includegraphics{Group_B9_files/figure-latex/unnamed-chunk-4-1.pdf}

The plot plot above looks linear inverse exponential, where there are a
lot of points near 0. This in comparision to the plot in 4.1 where it is
sort of a cloud of points where it is harder to see their relative
importance to each other.

\pagebreak
\# Code appendix

\begin{Shaded}
\begin{Highlighting}[]
\CommentTok{#setup}
\NormalTok{knitr}\OperatorTok{::}\NormalTok{opts_chunk}\OperatorTok{$}\KeywordTok{set}\NormalTok{(}\DataTypeTok{echo =} \OtherTok{FALSE}\NormalTok{, }\DataTypeTok{warning=}\NormalTok{F)}
\KeywordTok{library}\NormalTok{(MASS)}
\KeywordTok{library}\NormalTok{(readxl)}
\KeywordTok{library}\NormalTok{(knitr)}
\KeywordTok{library}\NormalTok{(tree)}
\KeywordTok{library}\NormalTok{(e1071)}
\KeywordTok{library}\NormalTok{(ROCR)}
\KeywordTok{library}\NormalTok{(fastICA)}

\CommentTok{#1.1}
\NormalTok{data1 =}\StringTok{ }\KeywordTok{read.csv}\NormalTok{(}\StringTok{"australian-crabs.csv"}\NormalTok{)}

\NormalTok{male_data =}\StringTok{ }\NormalTok{data1[data1}\OperatorTok{$}\NormalTok{sex }\OperatorTok{==}\StringTok{ "Male"}\NormalTok{,]}
\NormalTok{female_data =}\StringTok{ }\NormalTok{data1[data1}\OperatorTok{$}\NormalTok{sex }\OperatorTok{==}\StringTok{ "Female"}\NormalTok{,]}

\KeywordTok{plot}\NormalTok{(data1}\OperatorTok{$}\NormalTok{CL, data1}\OperatorTok{$}\NormalTok{RW, }\DataTypeTok{ylab=}\StringTok{"Rear Width"}\NormalTok{, }\DataTypeTok{xlab=}\StringTok{" Carapace length"}\NormalTok{)}
\KeywordTok{points}\NormalTok{(male_data}\OperatorTok{$}\NormalTok{CL, male_data}\OperatorTok{$}\NormalTok{RW, }\DataTypeTok{col=}\StringTok{"blue"}\NormalTok{)}
\KeywordTok{points}\NormalTok{(female_data}\OperatorTok{$}\NormalTok{CL, female_data}\OperatorTok{$}\NormalTok{RW, }\DataTypeTok{col=}\StringTok{"red"}\NormalTok{)}

\KeywordTok{legend}\NormalTok{(}\DecValTok{35}\NormalTok{, }\DecValTok{10}\NormalTok{, }\DataTypeTok{legend=}\KeywordTok{c}\NormalTok{(}\StringTok{"Male crabs"}\NormalTok{, }\StringTok{"Female crabs"}\NormalTok{),}
       \DataTypeTok{col=}\KeywordTok{c}\NormalTok{(}\StringTok{"blue"}\NormalTok{, }\StringTok{"red"}\NormalTok{), }\DataTypeTok{lty=}\DecValTok{1}\NormalTok{, }\DataTypeTok{cex=}\FloatTok{0.8}\NormalTok{)}

\CommentTok{#1.2}
\NormalTok{lda_crabs =}\StringTok{ }\KeywordTok{lda}\NormalTok{(sex }\OperatorTok{~}\StringTok{ }\NormalTok{CL }\OperatorTok{+}\StringTok{ }\NormalTok{RW, }\DataTypeTok{data=}\NormalTok{data1, }\DataTypeTok{CV=}\OtherTok{TRUE}\NormalTok{)}
\NormalTok{pred_female_data =}\StringTok{ }\NormalTok{data1[lda_crabs}\OperatorTok{$}\NormalTok{posterior[,}\DecValTok{1}\NormalTok{] }\OperatorTok{>}\StringTok{ }\NormalTok{lda_crabs}\OperatorTok{$}\NormalTok{posterior[,}\DecValTok{2}\NormalTok{],]}
\NormalTok{pred_male_data =}\StringTok{ }\NormalTok{data1[lda_crabs}\OperatorTok{$}\NormalTok{posterior[,}\DecValTok{2}\NormalTok{] }\OperatorTok{>}\StringTok{ }\NormalTok{lda_crabs}\OperatorTok{$}\NormalTok{posterior[,}\DecValTok{1}\NormalTok{],]}

\KeywordTok{plot}\NormalTok{(data1}\OperatorTok{$}\NormalTok{CL, data1}\OperatorTok{$}\NormalTok{RW, }\DataTypeTok{ylab=}\StringTok{"Rear Width"}\NormalTok{, }\DataTypeTok{xlab=}\StringTok{" Carapace length"}\NormalTok{)}
\KeywordTok{points}\NormalTok{(pred_male_data}\OperatorTok{$}\NormalTok{CL, pred_male_data}\OperatorTok{$}\NormalTok{RW, }\DataTypeTok{col=}\StringTok{"blue"}\NormalTok{)}
\KeywordTok{points}\NormalTok{(pred_female_data}\OperatorTok{$}\NormalTok{CL, pred_female_data}\OperatorTok{$}\NormalTok{RW, }\DataTypeTok{col=}\StringTok{"red"}\NormalTok{)}

\KeywordTok{legend}\NormalTok{(}\DecValTok{35}\NormalTok{, }\DecValTok{10}\NormalTok{, }\DataTypeTok{legend=}\KeywordTok{c}\NormalTok{(}\StringTok{"Male crabs"}\NormalTok{, }\StringTok{"Female crabs"}\NormalTok{),}
       \DataTypeTok{col=}\KeywordTok{c}\NormalTok{(}\StringTok{"blue"}\NormalTok{, }\StringTok{"red"}\NormalTok{), }\DataTypeTok{lty=}\DecValTok{1}\NormalTok{, }\DataTypeTok{cex=}\FloatTok{0.8}\NormalTok{)}

\NormalTok{pred_sex_list =}\StringTok{ }\NormalTok{lda_crabs}\OperatorTok{$}\NormalTok{posterior[,}\DecValTok{2}\NormalTok{] }\OperatorTok{>}\StringTok{ }\NormalTok{lda_crabs}\OperatorTok{$}\NormalTok{posterior[,}\DecValTok{1}\NormalTok{] }\CommentTok{# True is male}
\NormalTok{sex_list =}\StringTok{ }\NormalTok{data1}\OperatorTok{$}\NormalTok{sex }\OperatorTok{==}\StringTok{ "Male"} \CommentTok{# True is male}

\NormalTok{wrongs =}\StringTok{ }\KeywordTok{length}\NormalTok{(data1[sex_list }\OperatorTok{!=}\StringTok{ }\NormalTok{pred_sex_list, ][,}\DecValTok{1}\NormalTok{])}

\NormalTok{missclassification =}\StringTok{ }\NormalTok{wrongs }\OperatorTok{/}\StringTok{ }\KeywordTok{length}\NormalTok{(data1[,}\DecValTok{1}\NormalTok{])}

\NormalTok{missclassification}

\CommentTok{#1.3}
\NormalTok{lda_crabs =}\StringTok{ }\KeywordTok{lda}\NormalTok{(sex }\OperatorTok{~}\StringTok{ }\NormalTok{CL }\OperatorTok{+}\StringTok{ }\NormalTok{RW, }\DataTypeTok{data=}\NormalTok{data1, }\DataTypeTok{CV=}\OtherTok{TRUE}\NormalTok{, }\DataTypeTok{prior=}\KeywordTok{c}\NormalTok{(}\FloatTok{0.1}\NormalTok{, }\FloatTok{0.9}\NormalTok{))}
\NormalTok{pred_female_data =}\StringTok{ }\NormalTok{data1[lda_crabs}\OperatorTok{$}\NormalTok{posterior[,}\DecValTok{1}\NormalTok{] }\OperatorTok{>}\StringTok{ }\NormalTok{lda_crabs}\OperatorTok{$}\NormalTok{posterior[,}\DecValTok{2}\NormalTok{],]}
\NormalTok{pred_male_data =}\StringTok{ }\NormalTok{data1[lda_crabs}\OperatorTok{$}\NormalTok{posterior[,}\DecValTok{2}\NormalTok{] }\OperatorTok{>}\StringTok{ }\NormalTok{lda_crabs}\OperatorTok{$}\NormalTok{posterior[,}\DecValTok{1}\NormalTok{],]}

\KeywordTok{plot}\NormalTok{(data1}\OperatorTok{$}\NormalTok{CL, data1}\OperatorTok{$}\NormalTok{RW, }\DataTypeTok{ylab=}\StringTok{"Rear Width"}\NormalTok{, }\DataTypeTok{xlab=}\StringTok{" Carapace length"}\NormalTok{)}
\KeywordTok{points}\NormalTok{(pred_male_data}\OperatorTok{$}\NormalTok{CL, pred_male_data}\OperatorTok{$}\NormalTok{RW, }\DataTypeTok{col=}\StringTok{"blue"}\NormalTok{)}
\KeywordTok{points}\NormalTok{(pred_female_data}\OperatorTok{$}\NormalTok{CL, pred_female_data}\OperatorTok{$}\NormalTok{RW, }\DataTypeTok{col=}\StringTok{"red"}\NormalTok{)}

\KeywordTok{legend}\NormalTok{(}\DecValTok{35}\NormalTok{, }\DecValTok{10}\NormalTok{, }\DataTypeTok{legend=}\KeywordTok{c}\NormalTok{(}\StringTok{"Male crabs"}\NormalTok{, }\StringTok{"Female crabs"}\NormalTok{),}
       \DataTypeTok{col=}\KeywordTok{c}\NormalTok{(}\StringTok{"blue"}\NormalTok{, }\StringTok{"red"}\NormalTok{), }\DataTypeTok{lty=}\DecValTok{1}\NormalTok{, }\DataTypeTok{cex=}\FloatTok{0.8}\NormalTok{)}

\NormalTok{pred_sex_list =}\StringTok{ }\NormalTok{lda_crabs}\OperatorTok{$}\NormalTok{posterior[,}\DecValTok{2}\NormalTok{] }\OperatorTok{>}\StringTok{ }\NormalTok{lda_crabs}\OperatorTok{$}\NormalTok{posterior[,}\DecValTok{1}\NormalTok{] }\CommentTok{# True is male}
\NormalTok{sex_list =}\StringTok{ }\NormalTok{data1}\OperatorTok{$}\NormalTok{sex }\OperatorTok{==}\StringTok{ "Male"} \CommentTok{# True is male}

\NormalTok{wrongs =}\StringTok{ }\KeywordTok{length}\NormalTok{(data1[sex_list }\OperatorTok{!=}\StringTok{ }\NormalTok{pred_sex_list, ][,}\DecValTok{1}\NormalTok{])}

\NormalTok{missclassification =}\StringTok{ }\NormalTok{wrongs }\OperatorTok{/}\StringTok{ }\KeywordTok{length}\NormalTok{(data1[,}\DecValTok{1}\NormalTok{])}

\NormalTok{missclassification}

\CommentTok{#1.4}
\NormalTok{crab_glm =}\StringTok{ }\KeywordTok{glm}\NormalTok{(sex }\OperatorTok{~}\StringTok{ }\NormalTok{CL }\OperatorTok{+}\StringTok{ }\NormalTok{RW, }\DataTypeTok{data =}\NormalTok{ data1, }\DataTypeTok{family=}\KeywordTok{binomial}\NormalTok{())}
\NormalTok{glm_pred =}\StringTok{ }\KeywordTok{predict}\NormalTok{(crab_glm, }\DataTypeTok{type=}\StringTok{"response"}\NormalTok{)}

\NormalTok{male_glm =}\StringTok{ }\NormalTok{data1[glm_pred }\OperatorTok{>=}\StringTok{ }\FloatTok{0.5}\NormalTok{,]}
\NormalTok{female_glm =}\StringTok{ }\NormalTok{data1[glm_pred }\OperatorTok{<}\StringTok{ }\FloatTok{0.5}\NormalTok{,]}

\KeywordTok{plot}\NormalTok{(data1}\OperatorTok{$}\NormalTok{CL, data1}\OperatorTok{$}\NormalTok{RW, }\DataTypeTok{ylab=}\StringTok{"Rear Width"}\NormalTok{, }\DataTypeTok{xlab=}\StringTok{" Carapace length"}\NormalTok{)}
\KeywordTok{points}\NormalTok{(male_glm}\OperatorTok{$}\NormalTok{CL, male_glm}\OperatorTok{$}\NormalTok{RW, }\DataTypeTok{col=}\StringTok{"blue"}\NormalTok{)}
\KeywordTok{points}\NormalTok{(female_glm}\OperatorTok{$}\NormalTok{CL, female_glm}\OperatorTok{$}\NormalTok{RW, }\DataTypeTok{col=}\StringTok{"red"}\NormalTok{)}

\NormalTok{slope =}\StringTok{ }\KeywordTok{coef}\NormalTok{(crab_glm)[}\DecValTok{2}\NormalTok{]}\OperatorTok{/}\NormalTok{(}\OperatorTok{-}\KeywordTok{coef}\NormalTok{(crab_glm)[}\DecValTok{3}\NormalTok{])}
\NormalTok{intercept =}\StringTok{ }\KeywordTok{coef}\NormalTok{(crab_glm)[}\DecValTok{1}\NormalTok{]}\OperatorTok{/}\NormalTok{(}\OperatorTok{-}\KeywordTok{coef}\NormalTok{(crab_glm)[}\DecValTok{3}\NormalTok{]) }
\KeywordTok{abline}\NormalTok{(}\DataTypeTok{a=}\NormalTok{intercept, }\DataTypeTok{b=}\NormalTok{slope, }\DataTypeTok{col=}\StringTok{"green"}\NormalTok{)}

\KeywordTok{legend}\NormalTok{(}\DecValTok{33}\NormalTok{, }\DecValTok{10}\NormalTok{, }\DataTypeTok{legend=}\KeywordTok{c}\NormalTok{(}\StringTok{"Male crabs"}\NormalTok{, }\StringTok{"Female crabs"}\NormalTok{, }\StringTok{"Decision boundary"}\NormalTok{),}
       \DataTypeTok{col=}\KeywordTok{c}\NormalTok{(}\StringTok{"blue"}\NormalTok{, }\StringTok{"red"}\NormalTok{, }\StringTok{"green"}\NormalTok{), }\DataTypeTok{lty=}\DecValTok{1}\NormalTok{, }\DataTypeTok{cex=}\FloatTok{0.8}\NormalTok{)}

\NormalTok{glm_sex_list =}\StringTok{ }\NormalTok{glm_pred }\OperatorTok{>}\StringTok{ }\FloatTok{0.5} \CommentTok{# True is male}
\NormalTok{sex_list =}\StringTok{ }\NormalTok{data1}\OperatorTok{$}\NormalTok{sex }\OperatorTok{==}\StringTok{ "Male"} \CommentTok{# True is male}

\NormalTok{wrongs =}\StringTok{ }\KeywordTok{length}\NormalTok{(data1[sex_list }\OperatorTok{!=}\StringTok{ }\NormalTok{glm_sex_list, ][,}\DecValTok{1}\NormalTok{])}

\NormalTok{missclassification =}\StringTok{ }\NormalTok{wrongs }\OperatorTok{/}\StringTok{ }\KeywordTok{length}\NormalTok{(data1[,}\DecValTok{1}\NormalTok{])}

\CommentTok{#missclassification}
\CommentTok{#slope}
\CommentTok{#intercept}

\NormalTok{data =}\StringTok{ }\KeywordTok{read.csv2}\NormalTok{(}\StringTok{"NIRSpectra.csv"}\NormalTok{)}
\NormalTok{data_cp =}\StringTok{ }\NormalTok{data}
\NormalTok{data_cp}\OperatorTok{$}\NormalTok{Viscosity =}\StringTok{ }\KeywordTok{c}\NormalTok{()}
\NormalTok{res=}\KeywordTok{prcomp}\NormalTok{(data_cp)}
\NormalTok{lambda=res}\OperatorTok{$}\NormalTok{sdev}\OperatorTok{^}\DecValTok{2}
\CommentTok{#eigenvalues}
\NormalTok{lambda}
\CommentTok{#proportion of variation}
\NormalTok{calc =}\StringTok{ }\NormalTok{lambda}\OperatorTok{/}\KeywordTok{sum}\NormalTok{(lambda)}\OperatorTok{*}\DecValTok{100}
\KeywordTok{sprintf}\NormalTok{(}\StringTok{"%2.3f"}\NormalTok{, calc)}
\KeywordTok{sprintf}\NormalTok{(}\StringTok{"Feature 1 & 2 gives us a percentage of %2.3f"}\NormalTok{, calc[}\DecValTok{1}\NormalTok{]}\OperatorTok{+}\NormalTok{calc[}\DecValTok{2}\NormalTok{])}
\KeywordTok{screeplot}\NormalTok{(res)}

\KeywordTok{plot}\NormalTok{(res}\OperatorTok{$}\NormalTok{x[,}\DecValTok{1}\NormalTok{], res}\OperatorTok{$}\NormalTok{x[,}\DecValTok{2}\NormalTok{])}
\NormalTok{U=res}\OperatorTok{$}\NormalTok{rotation}
\KeywordTok{plot}\NormalTok{(U[,}\DecValTok{1}\NormalTok{], }\DataTypeTok{main=}\StringTok{"Traceplot, PC1"}\NormalTok{)}
\KeywordTok{plot}\NormalTok{(U[,}\DecValTok{2}\NormalTok{],}\DataTypeTok{main=}\StringTok{"Traceplot, PC2"}\NormalTok{)}
\KeywordTok{set.seed}\NormalTok{(}\DecValTok{12345}\NormalTok{)}
\NormalTok{ica =}\StringTok{ }\KeywordTok{fastICA}\NormalTok{(data_cp, }\DataTypeTok{n.comp =} \DecValTok{2}\NormalTok{)}
\NormalTok{w_prim =}\StringTok{ }\NormalTok{ica}\OperatorTok{$}\NormalTok{K }\OperatorTok\StringTok{ }\NormalTok{ica}\OperatorTok{$}\NormalTok{W}
\KeywordTok{plot}\NormalTok{(w_prim[,}\DecValTok{1}\NormalTok{], }\DataTypeTok{main=}\StringTok{"Traceplot, W1"}\NormalTok{)}
\KeywordTok{plot}\NormalTok{(w_prim[,}\DecValTok{2}\NormalTok{], }\DataTypeTok{main=}\StringTok{"Traceplot, W2"}\NormalTok{)}
\KeywordTok{plot}\NormalTok{(ica}\OperatorTok{$}\NormalTok{X[,}\DecValTok{1}\NormalTok{], ica}\OperatorTok{$}\NormalTok{X[,}\DecValTok{2}\NormalTok{])}
\end{Highlighting}
\end{Shaded}


\end{document}
